\documentclass[12pt]{article}
\usepackage{amsmath}
\usepackage{amssymb}
\usepackage{amsthm}
\usepackage{haskell}
\usepackage{hyperref}
\usepackage{booktabs}

\hypersetup{colorlinks=true}

\title{The Lambda C Compiler}
\author{Roland C. Dowdeswell}

\theoremstyle{plain}
\newtheorem{thm}{Theorem}[section]
\newtheorem{prop}[thm]{Proposition}
\newtheorem*{cor}{Corollary}
\theoremstyle{definition}
\newtheorem{defn}{Definition}[section]
\newtheorem{exmp}{Example}[section]

%%
%% make an & operator.
\DeclareMathOperator{\bitand}{\texttt{\&}}

\begin{document}
\maketitle
\begin{abstract}

Hmmm, well, I wanted to write a compiler in Haskell and C
seemed to be a decent language to do.

\end{abstract}

\clearpage
\tableofcontents
%% \listoftables
\clearpage

\begin{section}{Prerequisites}
This section will discuss and have references to material that
is assumed in the rest of the article.  This subsumes but is
not limited to:
\begin{enumerate}
\item		working knowledge of C, and
\item		working knowledge of Haskell.
\end{enumerate}
\end{section}

\clearpage
\begin{section}{Implementation}
\input{Cprogram.lhs}
\end{section}

%% \bibliography{EffManCIDR}
%% \bibliographystyle{plain}

\end{document}
